\documentclass[a4paper]{article}%

\usepackage[no-math]{fontspec}
\usepackage[quiet,nolocalmarks]{polyglossia}
\usepackage{chngcntr}
\usepackage{enotez}
\setenotez{
	list-name={الحواشي},
	totoc=section
}
\let\footnote=\endnote
\usepackage{hyperref}
\usepackage{xcolor}
\hypersetup{
  bookmarks=true,         % show bookmarks bar?
  unicode=true,          % non-Latin characters in Acrobat’s bookmarks
  pdftoolbar=true,        % show Acrobat’s toolbar?
  pdfmenubar=true,        % show Acrobat’s menu?
  pdffitwindow=true,     % window fit to page when opened
  pdfstartview={FitH},    % fits the width of the page to the window
  pdftitle={ما لا يمكنك قوله},    % title
  pdfauthor={Paul Graham},     % author
%  pdfsubject=,   % subject of the document
  pdfcreator=aborazmeh,   % creator of the document
  pdfproducer=aborazmeh, % producer of the document
%  pdfkeywords={}, % list of keywords
  pdfnewwindow=true,      % links in new PDF window
  colorlinks=true,       % false: boxed links; true: colored links
  linkcolor={red!50!black},          % color of internal links (change box color with linkbordercolor)
  citecolor={blue!50!black},        % color of links to bibliography
  filecolor=magenta,      % color of file links
  urlcolor={blue!60!black}           % color of external links
}

\setmonofont{Source Code Pro}
\defaultfontfeatures{Scale=MatchLowercase}

\newfontfamily\arabicfont[Script=Arabic, Scale=1.3]{Amiri}%
\newfontfamily\arabicfonttt[Script=Arabic,Scale=.75]{Source Code Pro}

\setdefaultlanguage[calendar=gregorian,hijricorrection=1,locale=mashriq,numerals=mashriq]{arabic}
\setotherlanguage[variant=british]{english}

\begin{document}

\title{ما لا يمكنك قوله}
\author{\textenglish{Paul Graham}\\
	\small{ترجمة \href{http://www.aboraz.me}{aborazmeh}}}
\date{كانون الثاني (يناير) 2004}
\maketitle
\tableofcontents
\clearpage

%% TODO:  image

هل حدث ورأيت صورةً لك قديمة فتحرّجت مما رأيت؟ ففكّرت: هل كنا حقاً نلبس هكذا؟ حسناً،
لقد كنا. ولم تكن لدينا أدنى فكرة عن سخف مظهرنا آنذاك. إن طبع الزي واللباس أن
يكون خفياً، بخفاء مسير الأرضِ الكوكبِ التي نحيا عليها جميعاً.

ما يذعرني وجود ما يُدعى بأخلاق الزيّ أيضاً. وهي مجموعة قواعد اعتباطية قد خفيت عن
معظم الناس، ولكنها أعظم خطراً إذ كما يدلّ الزي الأنيق على المصمم الفذ، يدلّ الزيّ
الأخلافي على خيرية متخذه. فاتخاذك زياً غريباً قد يؤدي بك أن توضع موضع السخرية،
والحيد عن هذه قواعد الزيّ قد يودي بك إلى الفصل من عملك، أو أن تنبذ في مجتمعك، أو
أن تسجن، أو حتى أن تقتل.

لو استطعت أن تسافر ماضياً في الزمن، فسيبقى شيءٌ -مما سيبقى- صحيحاً لن يتغير حيثما
ذهبت: وهو أنك عليك أن تحفظ لسانك. قد توقع بك أقوالاً نراها اليوم لا شيء في مهلكة
جليلة. بل قد مر مني ذكر واحدٍ على الأقل كان قد يودي بي في مشكلة عويصة في أي مكانٍ
كان من أوروبا في القرن السابع عشر، وقد جرت على جاليله فعلاً حينما سلف القول منه
بذلك =أعني إن الأرض تتحرك \footnote{لعل محاكم التفتيش لم يكن في نيتها آنذاك
  إمضاء وعيدها بالتعذيب. ولكن ذلك لأن جاليله كان قد أوضح جلياً أنه سيفعل ما أُمر
  به، ويصعب علينا افتراض تراجعهم إن بدر منه رفضٌ ما، ولم يكن قد مضى آنذاك عهدٌ
  بعيد على إحراق الفيلسوف جوردانو برونو إذ أصرّ هذا على عناده.}.

يبدو هذا مضطرداً خلال التاريخ: في كلّ حقبة، يعتقد الناس فيما غبر اعتقاداتٍ تماثل
هذا في سخفها، ويستمسكون بهذه الاعتقادات حتى يصير الإدلاء بعكسها يوردك المهالك.

هل اختلف زماننا هذا؟ لعلّ مجيباً قد قرأ شيئاً من التاريخ أن يجيب بشبه التأكيد أنْ
لا. ولعلها أن تكون صدفة عجيبة أن يكون عصرنا أول حقبة قد وضعت كل شيء في محله
الصحيح.

من المزعج التفكير بإمكان استسخاف الناس في المستقبل لبعض ما نعتقده اليوم. ما الذي
سيجب على المسافر بالزمن القادم إلينا من المستقبل أن يحترش من التفوه به؟ هذا ما
أردت دراسته هنا. ولكنني أرغب زيادة على أن أهزّ بعض النفوس بما قد يراه بدعة اليوم
أن أهتدي إلى السبيل العامة لمعرفة ما لا يمكنك قوله، في أية حقبة.

\section{اختبار الشخص التقليديّ}

لنباشر باختبار: هل ترى أنك تحتفظ في سريرتك بآراء تستهجن أن تسردها في جمع من
أقرانك؟

إذا قلت لا فلعلك تريد أن تقف وتراجع ما أسلفتَ للتو. فإذا كان جميع ما تعتقده مطابقُ
لما يُملى عليك وما يُراد منك اعتقاده، فهل يرجع هذا للصدفة المحضة؟ لعله ليس كذلك.
بل الأرجح أنك تفكر كيفما تُوجه.

أما الاحتمال الثاني فأن تكون قد استقللت بنفسك شرعت تنظر في هذه الآراء رأياً رأياً
وأجلت فيها الفكر فرجعت إلى هذه الاعتقادات التي اعتبرت صواباً في عهدك وحدث كل ذلك
اتفاقاً. ولا يبدو هذا رأياً راجحاً، لأنك لابد وأن تقع في الأغلاط على أقل تقدير.
يأتي رساموا الخرائط بالخطأ في خرائطهم عامدين حتى يطمئنوا إلى أن خرائطهم فريدة من
جهة وجود هذه الأغلاط، ويتبينوا بذلك الخرائط المنسوخة سرقةً، فإذا وجدوا الأغلاط
نفسها في خريطة لم يصنعوها بأنفسهم، كفاهم ذلك دليلاً أن الخريطة منسوخة وليست بعملٍ
أصليّ.

حقبتنا كأية حقبة من حقب التاريخ، ونكاد نوقن أن خريطتنا الأخلاقية فيها بعض
الأغلاط، وأن من يقع في الأغلاط هذه ذاتها لم يقع فيها جزافاً بمحض المصادفة. يشبه
هذا أن يزعم شخصٌ ما أن الناس عامَ 1972 قرروا فرادى من غير تأثر ببعضهم البعض أن
سراويلات الكم المنفوخ مفضلةٌ على غيرها.

وإذا كنت قد اعتقدت ما أُملي عليك إملاءً، فكيف لك أن تعرف أنك لن تتبع ما كان عليك
اعتقاده لو نشأت فرضاً إبان الحرب الأهلية الأمريكية في الجنوب؟ أو في المدة التي
تلت عام 1930 في ألمانيا النازية؟ أو بين المغول في 1200؟ بل لعل الأرجح أن تكون قد
اعتقدت ذلك.

سابقاً في عصور شيرع مصطلحاتٍ مثل "اعتقاد صالح" بدت هذه الفكرة وكأنك لست طبيعياً إن
فكرت بأفكار ولم تجرأ على الإفصاح بها على الملأ. يبدو هذا رجعياً، بل لعل الواقع
إنك على غير الطبيعة إن لم تفكر بما لا تقوى على الجهر به.

\section{مشكلة}

ما الذي لا يمكنك قوله؟ إحدى سبل معرفة هذه الأفكار النظرُ فيما يقوله الناس، ثم
يوردهم الموارد\footnote{تتكرم الكثير من المنظمات والمؤسسات بنشر لوائحً بالممنوع
  من القول داخل المؤسسة. ولكن تعاب هذه اللوائح بنقصها أولاً، إذا لدينا عُرفاً ما
  يشين حتى إن من يضع اللائحة لا يتخيل صدوره من أحد. وتُعاب ثانياً أنها توضع عامةً
  للغاية حتى ينعدم إمكان تطبيقها حرفياً. خذ مثلاً أنظمة الكلام في الجامعات إذا
  طُبقت حرفياً قلّما أن تسلم من أحدها نصوص شيكسبير.}.

لا يشمل بحثنا بالطبع كل ما لا يمكنك التفوه به، بل يختص بأشياء لا يمكننا القول
بصحتها، أو على الأقل لها حيثية من قبول صفحة الصواب يوجب معها بقاء البحث في
قضيتها مفتوحاً، لكن كثيراً مما يقوله الناس فيوقعهم فيما لا يحبون من القسم الثاني.
لا يقع في الجُرم من يزعم أن 2+2=5، أو أن البشر في بيتسبرغ أطوالهم عموماً ثلاثة
أمتار، بل تعامل أمثال هذه من المزاعم معاملة الطرائف، أو إذا ساء الأمر اعتُبرت
دليلاً على الجنون، ولكنها لا تبدو أنها تثير حنق أحد. إن التصريحات التي تغذب أحداً
ما من الناس هي تلك التي يخشون أن تُصدَّق. أظن أن أشد التصريحات إغضاباً للناس هي التي
يخشون كونها حقيقة.

لو أن جاليله زعم أن طول الناس في بادوفا ثلاثة أمتار، لاعتبر هذا مزعماً لا ضير
فيه، إلا أن زعم مدار الأرض حول الشمس قضية أخرى، فقد تفطنت الكنيسة إلى أن هذا لو
تُرك فسيدفع الناس لإعمال عقولهم.

وأثناء سعينا للبحث في التاريخ الماض، فإن هذا المبدأ العام يصدق بالطبع، فالكثير
مما عبّر عنه الناس فأوقعهم في الحرج أمس يُرى لا حرج فيه اليوم. ولهذا السبب نرجح أن
زائرنا من المستقبل سيوافق قولنا في ما يتحرج منه اليوم أو في بعضه على الأقل. هل
نفد الجاليلهات في زماننا؟ لا يبدو ذلك.

إذا طلبت أمثال جاليله في هذا الزمن، فما عليك إلا أن تطلب الآراء التي تحرج
أصحابها، ثم تورد عليها مسألة "أليس يحتمل صواب هذا الرأي؟". نعم، قد يكون هرطقةً أو
تجديفاً (أو ما يماثل ذلك في اصطلاح العصر)، ولكن ألا يحتمل على ذلك صوابه؟

\section{الهرطقة}

على أن ما سبق يترك مسائل بلا جواب. ماذا مثلاً لو أن حادثة الإشكال على فكرة ما لم
تحدث قط؟ ماذا لو أن مسألة كارثية الإثارة حتى إن أحداً لم يجرأ على الإفصاح عنها
على العلن؟ كيف يمكننا البحث في هذه أيضاً؟

سبيلنا في تحقيق هذا أن نتبع هذه الكلمة؛ الهرطقة. يبدو باستقراء التاريخ ظهور مثل
هذا الوسم للإشارة إلى بعض التصريحات لإلغائها قبل أن يتاح لأيٍ كان الفرصة أن يسأل
عن إمكان صحتها. "التجديف (blasphemy)" و"انتهاك الحرمات (sacrilege)" و"الهرطقة
(heresy)" كانت أمثال هذه الوسوم في التاريخ الغربي، وماثلها في العصور المتأخرة
قولهم "معيب (indecent)" أو "غير لائق (improper)" أو "غير أمريكي (unamerican)".
لقد كسرت شوكة هذه الكلمات اليوم، كما تنكسر شوكة مثيلاتها مع الزمن، بل أصبحت هذه
الكلمات اليوم تذكر غالباً في معرض السخرية، ولكنها كانت ذات قوة في زمنها.

كلمة "انهزامي" مثلاً لا أثر لها سياسي اليوم، ولكنها استعملت كسلاح في ألمانيا
عام 1917. استعملها لودندورف في حملات تطهير من فضّلوا السلام عبر المفاوضات. وكذلك
خصها بالاستعمال تشرشيل ومؤيدوه في بداءه الحرب العاليمة الثانية لإسكات الخصوم،
فقد وسم أي نقاش تناول سياسة تشرشيل العدائية بالـ"انهزامي". هل كانت صواباً أم خطأً؟
لا يبدو هذا اليوم محلاً للسؤال.

ولدينا اليوم بالطبع الكثير من هذه الوسوم، بدءً مما ينقد نقداً عاماً كـ "غير لائق"
إلى أكثرها إرهابات "مثير للانقسام". ينبغي على هذا أن تكون ملاحظة أمثال هذه
الوسوم في كل عصر من الأعصار سهلة متيسرة، فما عليك إلا أن تنظر ماذا يسم الناس
الأفكار التي تخالف معتقداتهم بعد سمة "خلاف الصواب". عندما يطلق سياسيٌ على خصمه
صفة الخطأ ، فهذا نقدٌ مباشر، أما عندما ينعته بعبارات من أمثال "مثير للانقسام" أو
"غير مراعٍ للأقليات أو الأعراق" مغضياً عن البحث في غلط منافسه، ينبغي علينا عند ذلك
أن نولي ذلك مزيداً من الأهمية.

ولهذا السبب قلنا إن لنا سبيلاً أخرى في تبيان من أي محرماتنا اللفظية سيضحك أبناء
الأجيال اللاحقة، وهذه السبيل أن نشرع في هذه الوسوم. خذ مثلاً وسم "جنسي (sexist)"،
وحاول أن تتفكر في بعض الأفكار التي ربما يشار إليها بذلك، ثم اسبرها واسأل نفسك
عند كل واحدة، ألا يحتمل صواب هذا القول؟

ابدأ فقط بالاستماع للآراء جزافاً؟ نعم، لأننا مذهبنا أن الأفكار لن تكون حقاً
عشوائية. إن أول ما يخطر إلى الذهن من الأفكار أصدقها، وهي هذه الأفكار التي طرقت
نظرك فلاحظتها ولكنك لم تتبعها فكرك.

تفطّن بعض الباحثين عام 1989 بمراقبتهم لعيون أطباء الأشعة أثناء فحصهم لصور للصدر
بحثاً عن سرطان الرئة\footnote{\textenglish{Kundel HL, Nodine CF, Krupinski EA,
    "Searching for lung nodules: Visual dwell indicates locations of
    false-positive and false-negative decisions." Investigative Radiology, 24
    (1989), 472-478.}}، إلى أن هؤلاء الأطباء يتوقفون برهة من الوقت عند موطن
العلة حتى عندما يفوتهم تشخيصها فيما بعد. إن قسيماً من عقلهم استنتج خللاً عند ذلك
المكان، ولكن ذلك لم ينفذ إلى وعيهم كل النفاذ. أظن أن كثيراً من خطرات الهرطقة قد
تشكل غالبها أساساً في عقولنا، فإذا أغلقنا عليها رقابتنا الذاتية مؤقتاً، فإنها ما
سيطفو أولاً.

\section{المكان والزمان}

لو أمكننا الاطلاع على المستقبل لعلمنا جلياً ما سيكون من ممنوعاتنا اللفظية مضحكاً
في ذلك الزمن. نحن لا يمكننا ذلك، إلا أننا يمكننا عمل أقرب شيءٍ يؤدي الغرض، أي
الاطلاع على الماضي، وصار لدينا طريقة أخرى للكشف عما نفهمه خطأً الآن، أن ننظر فيما
كان مقبولاً سابقاً وصار اليوم محرّماً.

تمثل التغيرات بين الماضي والحاضر أحياناً التقدم، فإذا خالفنا أسلافنا في ناحيةٍ
كالفيزياء مثلاً، فهذا لأننا على صواب، وهم كانوا على غلط، إلا أن هذا التأكيد على
الصواب والغلط يصير غير ممكناً عند الابتعاد عن الحقائق العلمية المؤكدة، وحينما تصل
إلى مجال الأسئلة الاجتماعية، تصير كثيرٌ من التغيرات كتغير الزيّ، فينضج قبول المرء
مجرى تغير طول الثياب.

لعلنا نظن بأنفسنا خيراً ونرانا أفطنَ وخيراً ممن سبقنا من الأجيال، لكن هذا ينقطع
شيئاً فشيئاً كلما قرأنا شيئاً من التاريخ. لقد كان أسلافنا كأمثالنا؛ لا أبطالاً ولا
همجاً. مهما كانت أفكار عصرهم، كانت بالنسبة إليهم أفكاراً مقبولةَ الاعتقاد لعموم
الناس.

فهذه سبيلٌ أخرى لترى مصدراً ممتعاً من مصادر الهرطقات. قارن (Diff) أفكار اليوم
بأفكار الثقافات الماضية المختلفة، فانظر ماذا ترى\footnote{\textit{استعمل الكاتب
    هنا مصطلح "\textenglish{diff}" وقال في الحاشية:} إن فعل "\textenglish{diff}" مصطلحٌ حاسبي، ولكنه كان
  الكلمة الوحيدة التي تؤدي غرض المعنى الذي أردت. لقد اشتُقت من أداة نظام Unix
  `\textenglish{diff}`، التي تطبع قائمة بجميع الاختلافات بين ملفين. عموماً قصدت بهذا المصطلح
  مقارنة شاملةً دقيقة، مجهرية ينتفي فيها كل نوعٍ من أنواع الانتقاء بين نوعين من
  الشيء المقارن.}. سيندهش بعض الباحثين من معيارات العصر الحاضر. حسناً ولكن ما
الذي أيضاً يحتمل الصواب؟

ليس لزاماً عليك أن تبحث فيما مضى لترى جلاء الاختلاف. إن للمجتمعات المختلفة في
عصرنا نظرات مختلفة جداً عما هو قبيحٌ وما هو حسن. ولذا يمكنك مقارنة (diffing) آراء الثقافات المختلفة بآراء ثقافتنا كذلك. (إن خير
ما تصنعه لذلك أن تزورهم في ديارهم).

لربما ستجد محرمَّاتٍ متضاربة. لربما تحريم التفكير في أمرٍ ما يستغرب في منطقة، بينما
تستغرب المسامحة فيه في أخرى، غير أنني أظن أن الاستغراب يكون غالباً في جهة واحدة،
بمعنى أنه يكون لا شيء فيه في ثقافةٍ ما، متعنتاً فيه في أخرى. أميل في هذا المجال
إلى القول بأن الجانب الذي يستثيره الرأي يكون غالباً الجانب الغلط\footnote{قد يبدو
  من هذا أنني مؤيّدٌ للأخلاقية النسبية (\textenglish{moral relativist}). هذا
  عكس الواقع، إنني أظن أنّ "حُكميّ (judgemental)" أحدُ الوسوم
  التي استُعملت في زماننا لكبح نقاش الأفكار ومحاولاتُنا أن نكون "لا حُكمييّن (non-judgemental)" ستبدو للأجيال اللاحقة إحدى أطرفِ
  أشيائنا.}.

يغلب على ظني أن مااشتدّ تحريمه كان محرماً عاماً بين جميع الثقافات، أو قريباً من
العام، كإزهاق الروح مثلاً. في حين إن وجود أي فكرة قد تعتبر غير ذات خطر في مدة،
وعلى امتداد مساحة، ذوي اعتبار، ثم كانت هذه الفكرة محرمةً في عصرنا أو بيئتنا، فهذه
قرينةٌ على وجود غلط ما من جهتنا.

لنضرب مثلاً على ذلك. وزعت جامعة هارڤرد - في أجلى مظاهر الإصلاح السياسي
(\textenglish{political correctness}) أوائل عقد الـ1990 - على كلياتها والعاملين
بها؛ منشوراً يقول - فيما قال - إنه من غير اللياقة أن تطري زيّ زميلٍ أو طالب. لا مزيد
من عباراتٍ أمثال "قيمصك جميل". أظن إن هذا المبدأ نادر الحدوث في ثقافات العالم،
ماضياً أو حاضر، فالأعم أن يغلبَ اعتبارُ إطراء زي أحدٍ من الأدب على اعتباره من
الجفاء. فإذا كان الأمر كذلك، واستجرتنا الافتراضات، فسنمثّل لزائرنا من المستقبل
بمحرماتٍ عليه لزاماً أن يجتنبها إن حصل وضبط ساعة آلته الزمنية على مدينة كامبردج،
بولاية ماساتشوستس الأمريكية، عام 1993.

\section{المتفيهقون - Prigs}

تأكيداً، لو امتلكت ذريتنا آلاتٍ زمنية فسيكتبون دليلاً منفصلاً خاصاً بكامبردج وحدها.
لم تزل كامبردج مكاناً ذا عناية خاصة، بلاد الخاء المعجمة والثاء المثلثة، حيث ستصحح
أفكارك ويصوب لحنُك في المجلس عينه. وهذا يقودني إلى سبيلٍ أخرى من سبل البحث عن
المحرمات؛ افتح رؤوس المتفيهقين، وانظر ما بداخلها.

إن رؤوس الأطفال مخازن جميع محرماتنا، وإن كنا نميل إلى الظن بأن أفكار الأطفال
ينبغي أن تكون ساطعة نقية. إننا أثناء نقلنا الأفكار إلى عقول الصغار، لا نقوم فقط
بتبسيطها لتناسب تطور أفكارهم، بل نقوم كذلك بتطهيرها مما نظن أنه يناسب ما نرى أن
على الطفل التفكير به\footnote{وهذا يجعل العالمَ متناقضاً بالنسبة للطفل، طالما إنهم
  يرون ما يخالف ما أخبروا به. فمثلاً، لم أتمكن إطلاقاً من فهم لماذا بدأ "الكشافون"
  البرتغاليون من شق طرقهم حول الساحل الإفريقي. في الواقع، لقد كانوا يطاردون
  العبيد.\\\textenglish{Bovill, Edward, The Golden Trade of the Moors, Oxford,
    1963.}}.

يمكنك قياس ذلك مصغراً في إطار الكلام البذيء. لقد شرع العديد من أصدقائي بتكوين
أسهرهم، ويحاول جميعهم اجتناب ألفاظٍ كـ"النكاح" أو "الخراءة" على مسمع الرضيع، خشية
أن يتلقفها الرضيع وتأخذ طريقها في كلامه أيضاً. لكن هذه الألفاظ جزء من اللغة،
ويستعملها البالغون كل الوقت. إذن إن الآباء يقدّمون فكرة غير دقيقة عن اللغة بعدم
استخدامها. لماذا يصنع الآباء ذلك؟ لأنهم لا يظنون إنه يناسب الأطفال أن يستعملوا
اللغة جميعها، إذ يعجبنا أن يبدو الأطفال أبرياء\footnote{لا يلبث الأطفالُ أن
  يتعلموا هذه الكلمات من رفقائهم، ولكنهم يعرفون أنهم ينبغي عليهم اجتنابها. يصير
  لدينا حالٌ أشبه ما يكون بملهاةٍ موسيقية، حيثُ يستعمل الآباء هذه الكلمات بين
  أقرانهم، ويحترزون عنها أمام أطفالهم، ويستعملها الأطفال مع أقرانهم، ويحترزون
  عنها أمام آبائهم.}.

بمثل ذلك، يقصد معظم البالغين عامدين إلى أن يعطوا صورةً مختلفة عن العالم لأطفالهم.
لعلّ أظهر مثال على ذلك "بابا نويل". نظن معشر الراشدين أنه يظرف ويلطف بالطفل أن
يصدق بـ"بابا نويل"، وأنا ذاتي أرى أنه يظرف بالطفل أن يصدق له. لكن يتساءل المرء،
هل نقول ما نقوله في مصلحة الأطفال أم في مصالحنا نحن؟

لا أريد هنا أن أناقش هذه الفكرة لا إيجاباً ولا سلباً. بل لعله مما لا محيد عنه أن
يرغب الآباء أن يلبسوا عقول أطفالهم قُمُط الرضّع. ولعلي في زمرة هؤلاء. ما يتصل
بغرضنا في هذا البحث هنا، نتيجةً لذلك، أن عقول الفتيان في بداية مراهقتهم تصير
مجموعةً من جميع محرماتنا جديدةً لم تستعمل بعد، وهذا لأنها لم تلوّث بعد بالخبرة
والممارسة. ما نستسخف ظهوره بعد ذلك ونرغب بكتمانه، ما زال يستخفي على الغالب في
تلك العقول.

كيف يمكننا إدراك مثل هذه الأفكار؟ يمكننا ذلك بتجربة فكرية. تخيّل آخر الأمر رجلاً
حلب الدهر أشطره، وخاض من الدنيا برها وبحرها، عملَ مدةً من عمره مبشراً في إفريقيا،
ولمدة طبيباً في نيبال، ومدةً مديراً لمقصفٍ ملهى ليلٍ في ميامي. دع عنك ما ذكرته من
المعيّنات وافترض رجلاً مارس وجرّب. ثم اسرح بخيالك وقارن ما ذا في رأس هذا الرجل
مقارنةً بفتاة مهذبة نشأت في ريفٍ من الأرياف. ماذا يظن هذا الرجل عما قد يذهل مثل
هذه الفتاة؟ إنه يعرف العالم، بينما تعرف هي -أو تتمثل على الأقل- محرمات بيئتها
وعصرها. اطرح أحدهما من الآخر، يبقى في يدك ما لا يمكننا قوله.

\section{المنهج}

أراني أستطيع إدراك سبيلٍ آخر لمعرفة ما يضيرنا إبداؤه: أن نكشف عن كيفية صناعة
المحرمات. كيف تظهر أزياء الأخلاق، وكيف تصنف؟ آذا استطعنا فهم هذه الكيفية، يمكننا
أن نراها تأتي ثمارها في عصرنا.

لا تنتشر الأزياء الأخلاقية بالطريقة ذاتها التي تنتشر بها الأزياء العادية. إذ
يبدو أن الأزياء تنتشر بالصدفة عندما يحاول الجميع محاكاة رغبةٍ طارئة لشخصٍ ذو
تأثير. لقد انتشرت أحذية المقدمة العريضة في أواخر القرن الخامس عشر لأنه كان
لتشارلز الثامن عشر ملك فرنسا ستة أصابع في إحدى قدميه. كما ظهرت أزياءُ باسم گاري
حينما تبنى الممثلُ [Frank Cooper](https://en.wikipedia.org/wiki/Gary_Cooper) اسم
مدينةً عُرفت بالشدة والصلابة في ولاية إنديانا. يظهر أن كثيراً من الأزياء تشيع على
عمد. حينما لا يمكننا الحديث عن مسألة، فهذا لأن جماعةً ما منعتنا عنها.

يبدو أن المنع يشتد عندما تكون الجماعة متعصبة تجاه شيء. الظريف في حال جاليله كان
أن ما ضرّه تكرار أفكار كوبرنيكوس. وبعكسه فعل كوبرنيكوس نفسه، بل كان هذا من أنصار
الكنيسة الباباوية، حتى إنه أهدى كتابه إلى اسم بابا روما. ولكن الكنيسة كانت في
عصر جاليله تعاني من نوبة مكافحة الإصلاح وكانت أكثر تزمتاً حول الأفكار البدعية.

لابدّ حتى يمضي التشنيع، أن تقف الجماعة موقفاً وسطاً بين الضعف والقوة. فصاحب السلطة
لا يحتاج أن يحمي محرماً، كما الحال في عصرنا أن لا ينظر إلى النيل من الأمريكان أو
الإنكليز على أنه غلط. كذلك لا بد للجماعة من أن يكون لها نصيبٌ من المنعة لتفرض
محرَّماً. إن الولعَ بالبَراز، كما هذا المقال، لا يبدو أنه يثير ما يكفي كماً ولا كيفاً
ليرقى أن يكون عاملاً مغيراً في أسلوب الحياة.

يغلب على ظني أن أعظم مصدر من مصادر المحرمات الأخلاقية سينقلب ليصير صراع سُلطة،
ولا يكاد يتفوق إحدى جانبي هذا الصراع على الآخر. هنا حيث ستجد جماعةً يتجاذبها
جانبي المنعة والضعف، منعةً تكتفي بها لفرض المحرم، وضعفاً يُحيجها إليه.

سيطمح معظم المناضلين أياً كان سببُ صراعاتهم إلى أن يظهروا بمظهر المناضل عن الفكرة.
كما كانت حركة الإصلاح الإنگليزي في عمقها صراعاً على الثروة والسلطة، لكنها انتهت
إلى التظاهر بالنضال في سبيل الحفاظ على أرواح المواطنين الإنگليز من تأثير روما
الهدّام. فمن السهل دفع الناس إلى المعركة في سبيل فكرة. ثم مهما كانت نتيجة المعركة
تلك، سيظلون يعتبرون أفكارهم ممتحنةً، كما لو أنّ الله أشار إلى قبوله لها عندما
اختارَ كسبَ خصمهم.

يطيب لمجتمعنا أن ينظر إلى الحرب العامة الثانية على أنها كانت انتصاراً للحرية على
الأنظمة الشمولية، متناسياً أن الاتحاد السوفيتي كان في جملة المنتصرين.

لا أزعم هنا أن المناضلين لا يكون دفاعهم عن الفكرة أبداً، ولكن أقول إنهم دائماً
سيَظهرون على أنهم يدفعون عن فكرة، وافق ذلك حقائقهم أم خالفها. فكما لن تجدَ في الزيّ
أشذّ من زيٍّ أُلغيَ آنفاً، كذلك لا تجدُ أشدّ خطأً من مبادئ خصمٍ هُزم قريباً. ما زالت الفنون
التمثيليّة (Representational art) تتعافى الآن من استحسان هتلر وستالين
كليهما\footnote{عملت منذ بضعة أعوام في انطلاقة مشروعٍ كانت علامته التجاريّة مكوّنةً
  من دائرةٍ حمراء يتوسّطها حرف V. لقد أعجبني تلك العلامة جداً. أتذكر أنني أخذت أفكر
  بعد أن بدأنا باستعمال تلك العلامة بمُدة، إنها حقاً لعلامةٌ توحي بالقوة؛ دائرةٌ
  حمراء. يُقال إن الأحمرَ أكثر الألوان بدائةً، على خلاف في ذلك، والدائرةُ أشد
  الأشكال أساسيةً. لقد أعطيا معاً انطباعاً بصرياً. لماذا لا تستعمل كثيرٌ من الشركات
  الأمريكية علامة الدائرة الحمراء للدلالة عليها؟ وَي، أجل...}.

على الرغم مما يبدو من اختلاف مصادر انبلاج الأفكار، عن مصادر شيوع الأزياء، لكن
أسلوب اتخاذهما مسلكاً حياتياً يبدو واحداً. إذ يُساق أوائل المتطبّعين بالطموح: أقوامٌ
يربؤون بأنفهسم لتتميز عن سائر القطيع. وحالما تستحكم البدع هذه، تَلحقهم باتخاذها
طائفةٌ هي أعظم عدداً، يدفعها إلى ذلك الخوف\footnote{الخوف أقوى القوّتين بمراحل.
  عندما أسمع أحداً يستعمل كلمة "gyp" أقول له مستحضراً الجديّة أن المرء ينبغي أن
  يتجنب هذه الكلمة لما توحي بالإهانة للغجر (Romani) (أي الـGypsies). إن المعاجم
  تخالفني في أصل اشتقاقها، غير أن الاستجابة معها دائماً تقريباً تكون امتثالاً
  رهيباً. هناك شيء في الأعراف، عُرف الزيِّ والفِكْر، أنه يسلُبُ ثقة المرء: فحينما
  يتعلّمون شيئاً جديداً، سيُحسّون أنه شيءٌ كان ينبغي عليهم أن يعلموه قبلاً.}. تتبنّى هذه
المجموعة البدعةَ لا رغبةً في الظهور، بل خوفاً من أن تبقى هي الناشزة.

فإذا رغبتَ إذن إلى الكشفِ عما لا يمكنك قوله، فانظر إلى سبيل التعوّد ثم قِس وتوقّع
ماذا سينتج عنه من محظور القول. أيّ الجماعات ذات مكِنةٍ لكن على غير اطمئنان، وأيّ
أفكار قد ترغب تلك إلى كبح جماحها؟ هل من فكرةٍ انتهت إلى أن ترتبط بالمذمّة عندما
فشلَ حاملها في صراعٍ قريب؟ لو فرضنا امرءً متطبّعاً بالطموح همّ بأن يميزَ نفسه عن
الأساليب السائرة (كوالديه مثلاً)، أيّ أفكارهم سيرمي إلى رفضها؟ ما الذي يخشى الأناس
ذوو الطبع التقليدي من قوله؟

لن تكشف هذه الوسائل عن جميع ما لن نتمكن من قوله. يمكنني تصوّر بعضها لم تنتج عن
صراعٍ قريب. كثيرٌ من محرّماتنا تجذّرت في الماضي. غير أنّ هذا النهج، مدعوماً بالوسائل
الأربع الماضية سيظهرُ عدداً لا بأس به من أفكارٍ محظورة.

\section{لماذا}

قد يتساءل البعضُ لم يُرغبُ في ذلك البحث؟ علامَ يُذهب طوعاً إلى مجال تلك الأفكار 
الخسيسة فينكزها؟ وفيمَ البحث تحت الصخور؟

وصنيعي ذلك أولاً لذات السبب الذي دفعني في صغري إلى النظر تحت الصخور: أعني 
الفضول المجرّد. وأنا طُلَعةٌ على الأخضّ إلى أيّ محظور. فالْأنظر وأقرر في شأني.

ثانياً، يدفعني إلى فعل ذللك بغضي من أن أكون على خطأ. فإذا كان من المحتملِ، على
غرار ما مضى من الأحقاب، أن نعتقدَ ما سيظهر سُخفه في تالي الأزمنة، فأرغب إلى أن
أعرف ما سيكون ذلك لأتجنب اعتقاده.

ثالثاً، أفعل ذلك لما يصحبه من فائدةٍ للدماغ. كي تتقنَ عملك تحتاج إلى عقلٍ قادرٍ على
سبر غور أي شأوٍ. وستحتاج خصوصاً إلى عقلٍ عادته أن يفحص حيثما ينبغي ألا يذهب.

تنشأ الأعمال العيمة غالباً من الأفكار التي فوّتها الآخرون، ولا أكثر تفويتاً من
أفكارٍ حُظر مجالها. خذ الاصطفاء الطبيعيّ على سبيل المثال، إنها فكرةٌ قريبة المتناول.
علامَ لم يُفكّر بها أحدٌ قبلاً؟ حسناً؟ إن جواب كل ذلك واضحٌ أتمّ الوضوح. فداروين ذاته
كان في غايةٍ من الحذر فيما يتعلّق بنظريته من الآثار والنتائج. لقد كانت همّته إلى
التفكير بيولوجياً، ليس إلى جدال من اتّهمه بالإلحاد من الناس.

إنها لميزةٌ عظيمةٌ في العلومِ الطبيعيةِ خصوصاً أن تكون قادراً على فحص المسلّمات. إن
منهج العلماء، أو الجيّدِ منهم على الأقلّ تحديداً هكذا: ابحث حيث الحكمة التقليدية قد
حُطّمت، ثم حاول نقبَ الحُطام وانظر ما تحته. فمن هنا تنبثق النظريات الجديدة.

بعبارةٍ أُخرى، إنّ عالماً جيداً لن يكتفيَ بتجاهل الحكمة التقليدية. بل سيرمي إلى بذل
المجهود في تحليلها. إنّ العلماء يجرون خلفَ المُشكلات. وينبغي أن يكون هذا منهج أيّ
بحّاثة. لكن يبدو على علماء الطبيعة أنهم أكثر ميلاً إلى النظر تحت
الصخور\footnote{لا أقصد الادّعاء أن آراء العلماء صائبةٌ دوماً، لكن أقول أن رغبتهم
  باعتبار الأفكار غير التقليدية يعطيهم دافعاً. بالنظر إلى جهةٍ أخرى قد يكون ذلك
  نقيصة. فكغيرهم من الباحثين، لا يكسبُ كثيرٌ من العلماء معاشاً أبداً-- أبداً، أي لا
  يكسبون لقاء أعمالٍ أنتجوها. يعيش معظم الباحثين في مجتمعاتٍ شاذة حيث يصلهم المالُ
  على هيئة صدقاتٍ بدل أن يكون تمثيلاً للعمل، ويبدو لهم طبيعياً أنه يجب على الاقتصاد
  القوميّ أن يتماشى مع ذلك. نتج عن ذلك أن توجّه كثيرٌ من اللامعين إلى الاشتراكية في
  منتصف القرن العشرين.}.

لم؟ قد يرجع ذلك باختصار إلى تفوقٍ في ذكاء العلماء؛ فمعظم الفيزيائيين لو اضطُرَّ الأمر
سينجزون ما يتطلبُ لنيل شهادة الدكتوراه في الآداب الفرنسية، غير أن قليلاً من أساتذة
الأدب الفرنسي قد يبلغون شهادة الدكتوراه في الفيزياء. أو لعل مرجع ذلك إلى الوضوح
في العلوم الطبيعية، حيثُ يظهر أي من النظريات صوابٌ وأيها خطأ، وهذا يصنع من العلماء
أشخاصاً أجرأ. (أو ربما تكون العلّةُ أنه بسبب وضوح النظريات في العلوم من حيث الصواب
والخطأ، سيكون عليك أن تكون أذكى ليكون عملُكَ في مضمار العلوم عوضاً أن تكون سياسياً ناجحاً).

مهما كان السببُ فإن الرابطة تتّضح بين الذكاء والإرادة على اعتبار الأفكار الصادمة.
ليس ذلك لأن الأذكياء يُلحّون في العمل على إيجاد الثغرات في الفكر التقليدي. بل أرى
أن التقاليد أصلاً ليس لها عليهم سلطانٌ كما لها على غيرهم. واعتبر ذلك أيضاً في
تزيّيهم.

لا يتوقف الأمر على العلماء حيث تُكسبهم بِدعُهم. ففي أيّ مجتمعٍ تنافسيّ، يمكنك أن تكسب
الكثير بأن ترى ما لا يراه غيرك. وفي كل مجال هناك غالباً بدعٌ يجرؤ قلّةٌ فقط على
التفوه بها. أُذيعت كثيرٌ من المخاوف في مجتمع صناعة السيارات في الولايات المتحدة
تتعلّق بتضاؤل حصتها من السوق. لكن السبب واضحٌ جداً لمن هو خارج نطاقهم، بحيث إنه
يشرح ذللك فيي ثانية واحدة: إنهم يُصنّعون سياراتٍ رديئة. ولقد طال أمد ذلك جداً، حتى
بلغ بهم الحال أن صارت أسماء شركات السيارت الأمريكية لوحدها منفّراتٍ للمشترين، بل
قد تشتري سيارةً لمجرّد كونها ليست من صنعهم. انقضى عهد كون الكاديلاك كاديلاك
السيارات منذ عام 1970. غير أنني أظن أنه يجرؤ على التصريح بهذا أحد\footnote{يُحتمل
  أن أفكاراً كهذه ستعتبر في المجتمع الصناعيّ سلبيّةً (negative). وقد توصم
  بالانهزامية (defeatist). لا تعبأ بذلك. على المرء أن يسأل، أصحيحة هي أم خطأ؟ إن
  مقياس صحة المؤسسة لعله يكون في درجة قَبولها بالأفكار السلبية. يبدو السلوك
  السائد حيث تُنتج الأعمال العظيمة دافعاً إلى النقد والسخرية، لا إيجابياً
  (positive) أو داعماً (supportive). من يقوم بالأعمال العظيمة ممّن أعرف، يظنون
  أنهم سيئون، بيدَ أن غيرَهم أسوء منهم.}. وإلا لكانوا سَعوا إلى إصلاح المشكلة.

يزيد النفع من التمرّن على التفكير بالمحظور على النفع من الأفكار ذاتها. إنه
كتمارين التمطيط. حينما تمطط عضلاتك قبل الجري، فإنك تجعل جسدك في وضعٍ أشدّ بكثير
مما سيلقاه أثناء الجري. ما إن يصير إطلاقُ فكرك عادةً تجعلُ شعورَ الناس تنتصب على
جلودهم عند سماعك، حينها لن يعوقك عائقٌ عن صغيرات الجولات التي يسمونها بالإبداع.

\section{Pensieri Stretti}

ماذا ستصنع بالمحظور القوليّ عند عثورك عليه؟ نصيحتي ألا تتفوه به، أو أن تختارَ
معركتك مليّاً على الأقل.

افترض أن حركةً قامت في المستقبل على منع اللون الأصفر. ستوصمُ أي دعوةٍ للرسم باللون
الأصفر بالـ"أصَفَرية"، وكذلك أياً يكن من يُشتبه في ميله إلى ذلك اللون. سيُسمح لمُحبيّ
البرتقاليّ غير أنهم سيكونون في مواقع المُراقبة. ثُمّ افترض أنك أدركت خلوّ الأصفر عن
المُشكلات، فإذا ذهبت تنشر هذا الرأي ستوصمُ أنك أصفريٌّ أيضاً، وستجدُكَ واقعاً في خضمّ
المُجادلات مع مناهضي الأصفريين. إن كانت غايتك من الحياة أن تعود بالسماح والاعتبار
إلى اللون الأصفر، فلعل تلك ستكون سبيلك الفُضْلى. ولكن إن كان أقصى اهتمامك إلى
أسئلة أُخرى، فإن الوصمة بالأصفرية ستكون مُجرّد إلهاء. جادلِ الأحمقَ، فتصيرَ مثله.

إن أكبر المهمّات أن تكون قادراً على أن تفكر فيما تريد، لا أن تقول ما تريد. ولعل
ظنك أنه ينبغي أن تقول جميعَ ما يجول بخاطرك يمنعُكَ من حوزة أفكارٍ مرضية. أظنّ أن
الأفضل أن تتبع المنهج المعاكس. خُطّ خطاً دقيقاً فاصلاً بين أفكارك وأقوالك. كل شيء
مسموحٌ داخل رأسك. فداخلَ رأسي أبلغُ مرحلةَ تشجيع أكثر الأفكار خطراً التي قدر خيالي
على استحضارها. ولكن، كما يحدث في التنظيمات السريّة؛ لا يُسمح بالإفصاح عن أي حدثٍ مما
يحدث داخل جدران البناء لمن هو خارجَه. أول قاعدةٍ من قواعد نادي القتال (يشير نادي
القتال \textenglish{Fight Club} إلى الفيلم الأمريكي المشهور.) ألا يُفصح أحدٌ عن
نادي القتال.

حينما أراد الشاعر ميلتون أن يزور إيطاليا في ثلاثينات القرن السابع عشر، أوصاه
النبيل الإنگليزي، الذي كان السفير الموفدَ إلى البندقية، السير هِنري ووتّون
(\textenglish{Henry Wootton}) أن يكون شعاره "\textenglish{i pensieri stretti \&
  il viso sciolto.}" =أفكارٌ محبوسة ووجهٌ مُنطلق. ابتسم لأيٍ كان، ولا تُفصح لهم عما
تفكر فيه. لقد كانت تلك نصيحة حكمة، إذ كان ميلتون ذا طبعٍ ميّالٍ إلى الجدال، وكانت
محاكم التفتيش مُتجهةً إلى بعضِ الجموح في ذاك الوقت. غير أنني أظن أن الاختلاف بين
حال ميلتون وحالنا ليس إلا اختلافاً في الدرجات. لكلّ عصرٍ هرطقاته، فإن لم يصل بك
الحال إلى السجن لأجلها، فستبلغ بك على الأقل إلى مشكلات تكفي أن تصيرَ مُشغلات تامّة.

إنني أقرّ أنه قد يبدو الصمتُ نوعاً من الجُبن. حينما أقرأ ما يصبّه أدعياء السينتولوجيا
من أذىً على منتقديهم\footnote{\textenglish{Behar, Richard, "The Thriving Cult of
    Greed and Power," Time, 6 May 1991.}}، أو ما ترتّبه الجماعات الداعمة لإسرائيل
من ملفاتٍ ضدّ من يصرّح بمخالفة الانتهاكات الإسرائيلية لحقوق
الإنسان\footnote{\textenglish{Healy, Patrick, "Summers hits 'anti-Semitic'
    actions," Boston Globe, 20 September 2002.}}، أو حول أناسٍ يُحاكمون لخرقهم
قانون الملكية الرقمية DMCA\footnote{\textenglish{"Tinkerers' champion," The
    Economist, 20 June 2002.}}، أرغبُ للحظاتٍ أن أقول: "حسناً، يا أولاد الحرام،
هاتوا ما عندكم". لكن المشكلة وجودَ عددٍ هائلٍ مما لا يمكنك قوله، فإذا بُحتَ به، لن
يبقى لك وقتٌ لتعمل فيه عملاً ذا فائدة، وستنقلبُ حينها إلى نَعوم
تشومسكي\footnote{أعني بذلك أنك ستصير جدليّاً بالمهنة، ليس أن آراء نعوم تشومسكي =ما
  لا يمكنك قوله. ستصدمُ لو قلتَ حقاً ما لا يمكنك قوله المحافظينَ والليبراليينَ على
  السواء تقريباً-- كما ستصدم أفكارك لو عُدت بآلة الزمن إلى إنگلترا الفيكتوريّة
  المحافظين (Tories) والأحرار (Whigs) تقريباً على السواء.}.

والمشكلة من الطرف الآخر أنك باحتفاظك بأفكارك في السر، ستفوّت ما في المناقشة من
الفوائد، إذا الكلام حول الفكرة، يولّد الفكرة. ينبغي أن تكون خُطّتنا المُثلى، أن نَحظى
عند الإمكان بعددٍ قليلٍ من الأصدقاء ممّن يُمكنك الحديث معهم حديثاً حُرّاً. ليست تلك
وسيلةً فقط لتطوير الأفكار، بل وأُسلوبٌ مُتّبع في اختيار الأصدقاء كذلك. إن لُقيا الأشخاص
الذين يمكنك التصريح بهرطقاتك لهم من غير أن يَثبوا عليك هي الأكثر إمتاعاً.

\section{\textenglish{Viso Sciolto?}}

لا أظننا سنحتاج انطلاق الوجه "\textenglish{viso sciolto}" على قدر ما احتجنا إسرار الفكر
"\textenglish{pensieri stretti}". لعلّ أقرب طريقة لذلك أن تكتفيَ بذكر رفضك لما قد تُعصّبَ له في
من غير تعيينٍ للذي دفعك إلى رفضه. سيحاول المتصّبون أن يفتنوك ولكنك غير مضطرٍ إلى
إجابتهم. فإذا حاولو أن يجرّوك إلى أن تواجه سؤالاً على شروطهم بقولهم "هل أنت معنا،
أم علينا؟" يمكنكَ الإجابة دوماً "لا معكم ولا عليكم".

تبقى الإجابة بـ "لم أقرر بعدُ" أفضل. ذلك ما فعله لاري سَمِّرز (\textenglish{Larry
  Summers}) حينما حاولت مجموعةٌ إيقاعه في ذلك الموضع. وقد شرح ذلك بعدُ بقوله "أنا
لا أختبِر باستخدام شرائح عبّاد الشمس"\footnote{\textenglish{Traub, James, "Harvard
    Radical," New York Times Magazine, 24 August 2003.}}. كثيرٌ من الأسئلة مما
يثيرُ حفيظة الناس ذو طبيعةٍ معقدة، ولا جائزة على الإجابة عنه بتلك السرعة.

إذا بدا أن مناهضي الأصفرية قد خرجوا عن السيطرة، وأردت أن تدافع، فهناك طرقٌ لذلك
من غير أن توصم بالأصفرية. كُن كالعصائب في العصور القديمة؛ اجتنب مواجهة قلب كتائب
العدوّ، وتوجه بنَبْلك إلى إزعاجها من بُعد.

إحدى السُبل إلى ذلك أن ترتفع بالمناظرة إلى مقام التجريد. فمُحاورتك ضد الرقابة على
العموم، سيُعفيك من التهمة التي قد تلحقك من الدفاع عن كتابٍ أو فيلمٍ قد يَرغب بعضهم
إلى أن تمنعه الرقابة. يمكنك محاورة الوسوم بوسوم الوسوم: أي الوسوم التي تُشير إلى
كيفية استعمال الوسوم، وذلك منعاً من الجدل. إن انتشار المصطلح "الإصلاح السياسي"
يعني بداءة نهاية الإصلاح السياسي، لأنه سيسمح للمرء أن يهاجِمَ الظاهرة برُمّتها من
غير أن يهتمّ بأي هرطقةٍ يُظنّ أن تنطوي عليها.

والمجازُ أيضاً وسيلةٌ أُخرى لردّ هجومٍ بمثله. حاولَ آرثر ميلر (\textenglish{Arthur
  Miller}) أن يقوّضَ لجنة التحقيق في الأنشطة المُعادية للأمركة (\textenglish{House
  Un-American Activities Committee (HUAC)}) بكتابة مسرحية "البوتقة
(\textenglish{The Crucible})" عن محاكمات السحر في سالم (\textenglish{Salem witch
  trials}). ولم يُشر إلى اللجنة تصريحاً فلم يَدع لهم وسيلةً للرد. ماذا عسى اللجنةَ أن
تصنع؟ أتدافع عن محاكمات السحر في سالم؟ عدا عن ذلك، ذاع مجازُ ميلر وبلغت به الشهرةُ
أن سُمّيت نشاطات اللجنة بـ"اصطياد المشعوذات (\textenglish{witch-hunt})".

لعل الفُكاهة أفضل الوسائل في النهاية. لا يكاد المتعصّبون على اختلاف مشاربهم يتحلّون
بخفّة الروح. ولا يتمكنون من الإجابة إلى الطرائف. مَثَلُ تعاسة هؤلاء في ميدان الطُرفة
كفارسٍ مُمتطٍ جواده على لوحِ تزلّج. يبدو أن هزيمة متحفّظي العصر الفيكتوري، على سبيل
المثال، كانت أساساً لأنهم عاملوها كطُرفة. كما تُقمّصت في الإصلاح السياسي. كتبَ آرثر
ميلر: "إنني راضٍ على كتابتي للبوتقة، غير أنني كثيراً ما أتمنى عندما أُعيدُ النظرَ لو
أعطيتها طابعَ ملهاةٍ عبثية، إذ ذلك ما يستحقه الواقع"\footnote{\textenglish{Miller, Arthur, The Crucible in History and Other Essays, Methuen, 2000.}}.

\section{كـ ـسـ ـد}

اقترحَ صديقٌ هولنديّ أنه ينبغي الاستشهاد بالمجتمع الهولندي مثالاً للتسامح الفكري.
إنه من الصحيح أن ذلك الإقليم أرّخَ لعصورٍ من الانفتاح العقلي نسبياً، حتى صار قرناً من
الزمانِ ملاذاً لكلّ مريدٍ من مواطني البلدان الواطئة ليُصرّحَ بشيء لا يمكنه إبرازه في
غيره. وأعانَ ذلك على أن يجعل البلدَ مركزاً للأستذة والصناعة (اللذَين كانا مترابطين
أكثر مما يظن كثيرٌ من الخلق). أنجز ديكارت معظم أفكاره في هولندة على الرغم من
ادعاء الفرنسيين له.

ومع ذلك مازلتُ أتساءل. لقد بدا أن الهولنديين عاشوا حيواتهم غارقين في الأصول
والتشريعات حتى آذانهم. هناك كثيرٌ مما لا يمكنك أن تفعله هناك،، فهل ينعدم حقاً ما
لا يمكنك قوله؟

أُأكّد هنا أن مجرّد تعظيمهم للانفتاح العقليّ ليس بشيء. مَن من الناس أصلاً يظنّ في نفسه
خلاف ذلك؟ إن آنستنا المُفْترضة من الضواحي لتظن الانفتاحَ العقليَّ من نفسها. ألم تُلقّن
أن تَظُنّ ذلك؟ اسأل أياً كان وسيجيبك الجواب نفسَه: إنهم ذوو عقليّةٍ منفتحة. بيد أنهم
يخطّون حول الأشياء بيّنة الغلط. (قد تعدو بعضُ القبائل كلمة "غلط" كوصفٍ حُكمي، إلى
كنايةٍ أكثر حياديّة كـ "سلبي" أو "هدّام").


عندما يكون البعضُ ضعفاء في الرياضيات فإنهم يعرفون ذلك، لعدم إصابتهم الأجوبةَ
الصحيحةَ في الاختبارات. لكن الناس لن يعرفوا من أنفسهم افتقارها للانفتاح العقليّ.
تذكّر أن من طبيعة الأعراف أن يخفى شيوعها، وبدون ذلك تفقد طبيعَتها. لا يبدو العُرف
عُرفاً لمن هو قابضٌ عليه، إذ يظهر له أنه ما يجبُ اتَباعه. ليست إلا سبيلَ الفحص من بُعْدٍ
تمكّننا من فصل الآراء الشخصية عمّا يجبُ اتّباعه، ومن ثَمَّ اعتبارِها أعْرافاً.


مرور الأيّام يعطينا هذا البُعْدَ مجاناً. حقاً إن تغيّر الأعراف إلى غيرها يجعل السابقات
سهلة التمييز، لأنها تبدو سخيفةً جداً بالقياس. في كلّ حركةٍ من حركات النوّاس، تبدو
النهاية المُقابلة أبعد ما يكون.

ملاحظة الشائع في عصرنا خصوصاً تحتاج إلى جهدٍ واعٍ. فمن غير الفاصلِ الزمنيِّ ستحتاجُ إلى
اصطناع بُعدك عن العصر بنفسك. بدلاً من الانفصال عن عُصبةٍ، قفْ أبعد ما تستطيع عنها
وانظر ماذا تفعلُ. واجهد لتُعملَ فكرك في الانتباه لأي أفكارٍ تنكتم. تتضمن مرشحات
الويب للأطفال والموظفين خياراتٍ لإلغاء الإباحيّة، والعُنف، وخطابات الكراهية. ما
الذي سيُعتبر إباحيةً وعُنفاً؟ وما "خطاب كراهية" بالتعريف؟ تبدو هذه جُملةً مُستلّةً من 1984.

لعلّ الوسوم أكبرُ الأدلّة الخارجية. فإذا كانت العبارة خطأً، ذلك أقصى ما ينبغي أن
توصم به، ولا عليك أن تصمها بالهرطقة. أما إن لم تكن خطأً فليس عليك أن تكبتها. فإذا
رأيتَ عباراتٍ تُهاجَمُ من مناصرين ومُعارضين على اختلافهم. سواءٌ في ذلك عام 1630 و2030،
سيكون ذلك علامة وجود خللٍ أكيد. عندما تسمعُ وسوماً كهذه تُستَعْمَل، اسأل لم.

خُصّ بالانتباه استعمالك لهم نفسك. ليست الجماعةُ فقط من تحتاج أن تراقبها من بُعد. يجب
أن تكون قادراً على مراقبة أفكارك من بُعْد. ليست هذه فكرةً جديدةً بالمناسبة. إنها
الفرق الأساسي بين الأطفال والراشدين. فعندما يغضبُ الطفل بسبب جوعه، فهو لا يدرك ما
الذي يحصل. أما الراشد فهو ينأى بنفسه عن الوضع بما يكفي ليقول "لا تهتمّ، إنما أنا
متعب". لا أُراني أجد سبباً صحيحاً لعجزِ المرءِ أن يُعمل نفس الوسيلة حتى يتعلّم أن يتعرّف
آثار الأخلاقيات الشائعة ويطرحها.

عليك أن تخطو تلكَ الخطوة إن أردتَ تفكيراً سليماً. لكنها أصعب، إذ أنت تعمل ضدّ
التقاليد الاجتماعية بدلاً من أن تجريَ معها. يُشجّعك الجميع لتنمو إلى المستوى حيث
يمكنك الإحاطة بطبائعك الرديئة. قليلون هم الذين يدفعونك إلى حيث يمكنك الإحاطة
بطبائع مجتمعك الرديئة.

أنّى يمكنك رؤية الموج، وأنتَ في عرض الماء؟ \textbf{كـُ}ـن
\textbf{سَـ}ـؤولاً \textbf{د}ائماً. هذا دفاعك الوحيد. ما الذي لا يمكنك قوله؟ ولم؟

\hrulefill

\textbf{يشكر} المؤلف كلاً من \textenglish{Sarah Harlin}، \textenglish{Trevor
  Blackwell}، \textenglish{Jessica Livingston}، \textenglish{Robert Morris}،
\textenglish{Eric Raymond}، \textenglish{Bob van der Zwaan} لقرائتهم مسودات هذا
المقال، ويشكر \textenglish{Lisa Randall}، \textenglish{Jackie McDonough}،
\textenglish{Ryan Stanley}، \textenglish{Joel Rainey} لمناقشتهم موضوع الهرطقة.
ويذكر عَدم تحمّلهم أي لوم تجاه الآراء المذكورة فيه، وخصوصاً تجاه الآراء التي لم
تُذكر فيه.

\printendnotes
\end{document}
